\documentclass{article}

\title{Noise Model for Imaging Simulator Software}

\author{John Young}

\begin{document}

\maketitle

\section{Introduction}

The noise model consists of formulae for the variance of the
squared visibility and bispectrum. These formulae are given in the
memo ``SNR squared visibility and closure phase: summary of equations
in use'' by Tatulli.

That memo should be read in conjunction with this note, which lists
parameter values appropriate for MROI. These parameter values should
be substituted into the expressions given by Tatulli.

\subsection{Assumptions}

The expressions given by Tatulli are not specific to any particular
beam combiner design, but are based on the following assumptions:
\begin{itemize}
\item Photon and detector noise
\item No background noise (e.g. from thermal radiation)
\item No atmospheric noise
\item The fringe signal is debiased \emph{a posteriori}
\item No calibration error (could add this, e.g.\ 2\% in $V^2$)
\end{itemize}

\section{Bispectrum noise}

The memo by Tatulli gives the variance of the closure phase but not
that of the triple (bispectrum) amplitude $T_{ijk} = |Q_{ijk}|$. Photon and
detector noise give rise to a circular noise cloud in the complex
plane, hence:
\begin{equation}
\sigma^2\{T_{ijk}\} = T^2_{ijk} \sigma^2\{\phi_{ijk}\} ,
\end{equation}
with $\sigma^2\{\phi_{ijk}\}$ in radians.

\section{Parameters}

The expressions given by Tatulli are in terms of the overall squared
visibility $|V^2|$ and number of photons detected per coherent
integration $\overline{N}$. The squared visibility can be calculated
as follows:
\begin{equation}
|V^2| = | \left( V_\mathrm{obj} V_\mathrm{inst} \right) |^2  ,
\end{equation}
where $V_\mathrm{obj}$ is the complex visibility calculated from the
object model (pixellated image), and $V_\mathrm{inst}$ is the
instrumental visibility degradation (see Table~\ref{tab:params}).

The photon count per coherent integration is given by:
\begin{equation}
\overline{N} = F_0 10^{-0.4m_\lambda} t_\mathrm{int} N_\mathrm{tel} \pi R^2
\Delta\!\lambda T S  ,
\end{equation}
where $m_\lambda$ is the apparent magnitude of the object at the
observation wavelength, and the other symbols are defined in
Table~\ref{tab:params}.

\begin{table}[p]
\caption{\label{tab:params} Parameter values to use.}
\begin{tabular}{ccll}
\hline
Param & Units & Meaning & Value\\
\hline
$N_\mathrm{pix}$ & & No.\ of pixels to sample fringes & 128 \\
$N_\mathrm{tel}$ & & No.\ of telescopes combined & Max. 4\\
$\sigma$ & e & Detector readout noise & 5e \\
$V_\mathrm{inst}$ & & Instrumental visibility & 0.67 \\
$R$ & m & Telescope aperture radius & 0.7m \\
$T$ & & Transmission (including detector QE) & 0.23 \\
$F_0$ & photon m$^{-2}$ s$^{-1}$ $\mu$m$^{-1}$
 & Photon flux from 0-mag star & Table~\ref{tab:wav_dep} \\
$S$ & & Strehl ratio & Table~\ref{tab:wav_dep} \\
$t_\mathrm{int}$ & s & Coherent integration time & Table~\ref{tab:wav_dep} \\
\hline
\end{tabular}
\end{table}

\begin{table}[p]
\caption{\label{tab:wav_dep} Values of wavelength-dependent
  parameters. These may be assumed constant within each waveband.}
\begin{tabular}{ccccc}
\hline
Band & Wavelength & Flux & Typical & Typical \\
name & range & / photon m$^{-2}$ s$^{-1}$ $\mu$m$^{-1}$
 & Strehl $S$ & $t_\mathrm{int}$ / s\\
\hline
J & 1.2--1.4 & $2.02 \times 10^{10}$ & 0.84 & $2.2 \times 10^{-2}$ \\
H & 1.5--1.8 & $9.56 \times 10^{9}$  & 0.95 & $2.9 \times 10^{-2}$ \\
K & 2.0--2.4 & $4.53 \times 10^{9}$  & 0.99 & $4.1 \times 10^{-2}$\\
\hline
\end{tabular}
\end{table}

\end{document}
